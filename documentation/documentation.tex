%package obligatoire : type de document
\documentclass[a4paper,12pt,twoside]{book}

% encodage
\usepackage{fontspec}

% le package hyperref avec des options, si en local
\usepackage[pdfusetitle, pdfsubject ={Mémoire TNAH}, pdfkeywords={les mots-clés}]{hyperref}

%il faut mettre au moins une langue
\usepackage[english,french]{babel}

% configurer le document selon les normes de l'école
\usepackage[margin=2.5cm]{geometry} %marges
\usepackage{setspace} % espacement qui permet ensuite de définir un interligne
\onehalfspacing % interligne de 1.5
\setlength\parindent{1cm} % indentation des paragraphes à 1 cm


% bibliographie
\usepackage[backend=biber, sorting=nyt, style=enc,maxbibnames=10]{biblatex}
\addbibresource{biblio.bib}
%\nocite{*}


% DOCUMENT
\begin{document}
	
	\part{HTR}
	
	\chapter{Un chapitre}
	
	\section{Présentation matérielle du mémoire}
	
	Recto verso. Taille 12, en police Time New Roman ou équivalent. Le texte doit être justifié ; les marges sont de 2.5 cm. L'interligne doit être fixé à 1.5 ; les paragraphes indentés à 1 cm. Toute citation longue, de plus de trois lignes, doit apparaître en retrait, sans guillemet (environnement \texttt{quotation} en \LaTeX ).
	
	\chapter{Un autre chapitre}
	
	\section{Structuration du mémoire}
	
	Le mémoire se structure en plusieurs parties :
	\begin{enumerate}
		\item tout d'abord, les pièces liminaires : page de titre, résumé, remerciements, bibliographie, introduction
		\item ensuite, le corps du texte, suivi d'une conclusion
		\item après, les annexes (documentation, extraits de code, etc.)
		\item enfin, les pièces finales : index (si besoin), glossaire (si besoin) ; tables (des figures et des tables, si nécessaire) ; table des matières
	\end{enumerate}
	
	Ce mémoire s'accompagne d'une autre partie très importante : les \textbf{données}.
	
	\section{Les données}
	
	Elles constituent une partie primordiale du travail à rendre. Ce sont :
	\begin{itemize}
		\item les données traitées (textes, images, vidéos, BDD, etc.)
		\item les scripts de traitement
		\item la documentation associée aux données et au script
		\item tout autre document qui semble nécessaire au traitement du sujet.
	\end{itemize}
	
	Ces données doivent être ordonnées et accompagnées d'un fichier \texttt{lisezMoi} (format \texttt{.txt} ou \texttt{.md}), présent à la racine du dossier contenant les données. Ce fichier doit décrire l'arborescence des fichiers et dossiers et la fonction de chacun des fichiers.
	
	Le principe important à retenir est celui de la \textbf{reproductibilité} du travail.
	
	\part{Une autre partie}
	
	%etc.
	
	
	\chapter*{Conclusion}
	\addcontentsline{toc}{chapter}{Conclusion}
	
	%les annexes
	\appendix
	\chapter{Première annexe}
	
	\backmatter
	
	% index à mettre ici si index	
	%	\printindex
	
	%glossaire si glossaire
	%	\printglossaries
	
	% si figures
	%	\listoffigures
	
	%si tables
	%	\listoftables
	
	\tableofcontents
	
\end{document}